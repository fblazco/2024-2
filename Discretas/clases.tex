
\documentclass{article}
\author{fblazco}
\title{IIC2233}
\usepackage{amsmath}
\usepackage{graphicx}
\begin{document}
	\maketitle
    \section{Clase 1}
    \begin{enumerate}
        \item \textbf{Fecha Evaluaciones}
    \end{enumerate}    
          
    \section{Clase 3}
    \subsection{Induccion Estructural}
        \textbf{Clase anterior: Teorema}
        \begin{enumerate}
            \item Principio del buen orden
            \item Principio de induccion simple
            \item Principio de induccion fuerte
        \end{enumerate}
        \textbf{Definiciones Inductivas}:
        Para definir inductivamente un conjunto necesitamos
        \begin{enumerate}
            \item Un conjunto de elementos base no necesariamente finito
            \item Un conjunto finito de reglas de construccion de nuevos elementos del conjunto a partir de elementos que ya estan en elementos
            \item Establecer que el conjunto es el menor que cumple las reglas
        \end{enumerate} 
        \textbf{Ejemplo}:
        El conjunto de numeros pares es el menor conjunto tq
        \begin{enumerate}
            \item El 0 siempre es pares
            \item Si n es un numero par, n+2 es numero par
        \end{enumerate}
        \textbf{Definicion (listas enlazadas) $\mathcal{L}$n}: El conjunto $\mathcal{L}$n es el menor conjunto que cumple con las sgts. reglas
        \begin{enumerate}
            \item $\phi$ $ \epsilon $ $\mathcal{L}$n
            \item Si L $\epsilon $ $\mathcal{L}$n y k $\varepsilon $ N, entonces L $\rightarrow$ k $\epsilon $ Ln
        \end{enumerate}
        \textbf{Cuando dos listas enlazadas son iguales?}

            \begin{enumerate}
                \item Si alguna es $\phi$, son iguales ssi la otra tmb es vacia
                \item Si ninguna es vacia entonces estamos en un escenario 
                \begin{center}
                    $L_{1}$$\rightarrow$ $k_{1}$ vs L2 $\rightarrow$ k2
                \end{center}
                en este caso, resulta natural considerar 
                \begin{center}
                    L1 $\rightarrow$ k1 = L2 $\rightarrow$ k2 ssi. L1=L2 y k1=k2
                \end{center}
            \end{enumerate}
        \subsection{Principio de Induccion Estructural}
        Sea A un conjunto definido inductivamente y P una propiedad sobre los elementos de A.
        Si se cumple que:
        \begin{enumerate}
            \item Todos los elementos base de A cumplen la propiedad P
            \item Para cada regla de construccion, si la regla se aplica sobre elementos
            en A que cumplen la propiedad P 
        \end{enumerate}
        \textbf{Ejemplo}
        P(L): L tiene el mismo numero de flechas que de elementos
        \begin{center}
            BI: El unico caso base de es la lista vacia
            \newline
            HI:
        \end{center}(terminar demostracion)
        \newline
        \textbf{Para demostrar propiedades en Ln definiremos mas operadores}(agregar operadores)
        \begin{enumerate}
            \item Largo, recibe lista y enetrega numero de elementos
            \item Suma, recibe lista y entrega la suma de sus elementos
            \item Maximo, recibe lista y entrega el maximo, o -1 si es vacia \begin{center}
                max: $L_{n}$ $\rightarrow$ N $\vee$ \{-1\}
            \end{center}
            \item Cabeza, recibe una lista no vacia y entrega su primer elemento \begin{center}
                head: $L_{n}$ $/$ \{$\phi$\} $\rightarrow$ $L_{n}$ 
            \end{center}
            Si k $\epsilon$ N, entonces suf($\rightarrow$k) = $\phi$ \newline
            Si 
        \end{enumerate}
        \textbf{Teorema (props, listas)}: Si L, L1, L2 $\epsilon$ Ln, entonces:
        \begin{enumerate}
            \item sum(L) $>=$ 0
            \item max(L) $<=$ sum(L)
        \end{enumerate}
        \textbf{Teorema (prop. 4 de listas)}
        Sean, L1, L2 $\epsilon$ Ln. Si L1, L2 $\neq$ $\phi$ (terminar)
        \newline
        \textbf{Demostracion:}
        Sea \begin{center}
            L1 =$\rightarrow$ K1 \\
            L2 =$\rightarrow$ K2 \\
            tq suf(L1) = suf (L2) \\
            sum(L1) = sum(L2)
        \end{center}
        Por definicion de igualdad de listas 
        \begin{center}
            L1=L2 \\
            $\phi$ $\rightarrow$ k = $\phi$ $\rightarrow$ y \\
        $\phi$ = $\phi$ \\
        y k= j
        \end{center}
        
\end{document}  

