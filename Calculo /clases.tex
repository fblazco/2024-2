
\documentclass{article}
\author{fblazco}
\title{MAT1610}
\usepackage{amsmath}
\begin{document}
	\maketitle
    \section{Clase 1}
        \subsection{Fechas evaluaciones}
            \begin{enumerate}
                \item  I1 lunes 2 de Septiembre (hasta Clase 10). Bloque 7-8
                \item  I2 Martes 8 de Octubre  (hasta Clase 20). Bloque 7-8
                \item  I3 Lunes 4 de Noviembre (hasta Clase 30). Bloque 7-8
                \item  Examen Lunes de Diciembre (toda la materia)
            \end{enumerate}
        \subsection{Ponderaciones}
            \begin{center}
                NF $\equiv$ Interrogaciones 20\%x3 + Lab 10\% + Ex 30\% 
            \end{center}
        \subsection{Limite de Funciones}
            \begin{enumerate}
                \item Consideremos esta funcion 
                \begin{equation}
                    f(x) = \frac{x^2}{x-1} \ , Dom(f) \rightarrow R -\{1\}
                \end{equation} Que ocurre si me acerco a 1?
                \item \begin{tabular} {c c}
                    $x\rightarrow 1- $& me acerco por la izquierda \\
                    $x\rightarrow 1+$ & me acerco por la derecha
                \end{tabular}
                \item si x $\rightarrow$ 1 $\longleftrightarrow$ f(x) $=$ ?
                \item esto lo podemos escribir como
                \begin{equation}
                    \lim_{x->1} f(x)=2
                \end{equation}
                \item tecnicamente: "si x esta muy cerca de 1 entonces f(x) esta muyt cerca de 2"
                \begin{equation}
                    f(x) = \frac{x^2-1}{x-1}= \frac{(x-1)\cdot (x+1)}{(x-1)} = x+1 
                \end{equation}
                \item 
                \begin{equation}
                    f(x) = x+1, x\neq 2
                \end{equation}
                \item Formalmente:
                \begin{enumerate}
                    \item Supongamos que f(x) esta definida para todo x cerca de "a" 
                    \item Escribiremos: \begin{equation}
                        \lim_{x->a} f(x) = L
                    \end{equation} Si cuando x se aproxima a "a" entonces f(x) se aproxima a "n"
                \end{enumerate}
                \item  Sea \begin{equation}
                    f(x) \equiv (x+2)^2 + 1, x<=-1   
                    f(x) \equiv (x+3), x>-1 
                    \end{equation}
                    Grafique f(x) y concluya el valor de \begin{equation}
                        \lim_{x->-1} f(x)
                    \end{equation} Si me acerco por + se acerca a 2 y si que acerco por - tambien me acerco a 2 (viendolo desde el grafico de la funcion)
                    y eso es igual al grafico de f(-1)
                \item Sea \begin{equation}
                    \begin{aligned}
                        f(x) \biggl\{    (x+2)^2+1,\ x\leq -1   
                        \\     
                        (x+3), \ x>-1 
                        \\
                    \end{aligned}
                \end{equation}
            \end{enumerate}
        \section{Clase 2}
            \subsection{Calculo de Algunos Limites}
            \begin{enumerate}
                \item \begin{equation}
                    \lim_{x->9} \frac{\sqrt[2]{x}-3}{x-9} =
                    \lim_{x->9} \frac{\sqrt[2]{x}-3 \cdot (\sqrt[2]{x} +3)}{x-9 \cdot (\sqrt[2]{x} +3)}  =
                    \lim_{x->9} \frac{\sqrt[2]{x}-3}{x-9} \equiv \frac{1}{6}
                \end{equation}
                \item \begin{equation} \lim_{x->2} \frac{x^2-x-2}{x-2} \equiv 3
                \end{equation}
                \item \begin{equation}
                    \lim_{x->-1} \frac{x^3 +1}{x+1} = 
                    \lim_{x->-1} \frac{(x+1)\cdot (x^2-x+1)}{(x+1)}=
                    \lim_{x->-1} (x^2-x+1) \equiv 3
                \end{equation}
                \item  \begin{equation}
                    \lim_{x->27} \frac{\sqrt[3]{x}-3}{x-27} =
                    \lim_{x->27} \frac{\sqrt[3]{x}-3}{(\sqrt[3]{x}-3) \cdot (\sqrt[3]{x}^2 + 3\sqrt[3]{x}+9)} =
                \end{equation}
                \item  \begin{equation}
                    \lim_{t->0} \frac{\sqrt[2]{t^2 +9}-3}{t^2} \equiv \frac{1}{6}
                \end{equation}
            \end{enumerate}
            \subsection{Limites Laterales}
            \begin{enumerate}
                \item \begin{equation}
                    \lim_{x->a+} f(x) = L2
                \end{equation}
                Limite lateral derecho y se da cuando x se acerca a "a" por la derecha
                \item \begin{equation}
                    \lim_{x->a-} f(x) = L1
                \end{equation}
                Limite lateral izquierdo y se da cuando x se acerca a "a" por la izquierda
                \item \begin{equation}
                    \lim_{x->a} f(x)
                \end{equation}\begin{center}
                    \textbf{No existe}
                \end{center}
            \end{enumerate}
            \textbf{Teorema}
             \newline   
            \begin{equation}
                    \lim_{x->a} f(x) =L \equiv \lim_{x->a+} f(x) = L = \lim_{x->a-} f(x)
            \end{equation} Obs: Limites laterales distintos imploca que el limite no existe
            \newline \textbf{Ejemplo:}
            \begin{enumerate}
                \item Para \begin{equation}
                    f(x) = \frac{x-1}{|x-1|}
                \end{equation} Calcular
                \begin{equation}
                    \lim_{x->1} f(x)
                \end{equation} 
                \begin{equation}
                    y=|x-1| \Biggl\{ 
                    \begin{aligned}
                          && (x-1) && ,si\ x>=1 \\ 
                          &&-(x-1) &&,si\ x<1
                    \end{aligned}
                \end{equation} Veamos los limites laterales de f(x)
                \begin{equation}
                    \lim_{x->+1}f(x)=\lim_{x->+1} \frac{x-1}{x-1} \equiv 1
                \end{equation}
                \begin{equation}
                    \lim_{x->-1}f(x)=\lim_{x->-1} \frac{x-1}{-(x-1)} = \lim_{x->-1} -1 \equiv -1
                \end{equation}
                -> \begin{equation}
                    \lim_{x->1} f(x)
                \end{equation} \begin{center}
                    NO existe pues los laterales son distintos
                \end{center}
                \item Sea g(x) = \begin{equation}
                    \biggl\{\begin{aligned}
                        2x && ,si\ x>=3 \\ 
                        \frac{x^2 -9}{x-3} &&,si\ x<3 
                    \end{aligned}
                \end{equation}
            \end{enumerate}
    \section{Clase 3}
    

    \section{Clase 4}
    \begin{enumerate}
        \item 
        \begin{equation}
            \lim_{x\rightarrow-2} f(x) + 5 \lim_{x->-2}g(x)
        \end{equation}
        \begin{equation}
            2 \cdot 1 + 5 \cdot (-2) - 10 = -8
        \end{equation}
        Finalmente
        \begin{equation}
            \lim_{x->2} f(x) = -8
        \end{equation}
        \item \begin{equation}
            \frac{\lim_{x\rightarrow1}f(x)}{\lim_{x\rightarrow1 f(x)- \lim_{x\rightarrow1}g(x)}}
        \end{equation}
        \begin{equation}
            \begin{aligned}
                  \lim_{x\rightarrow1-}g(x)=0 && \lim_{x\rightarrow1}g(x)\  NO \ existe \\ 
                \lim_{x\rightarrow1+}g(x)=-2&&
            \end{aligned}
        \end{equation}

    \end{enumerate}
    \textbf{Calculo de limites}:
    Evaluar
        \begin{equation}
            \frac{P(x)}{Q(x)} = \frac{\liminf P(x)}{\liminf Q(x)}
        \end{equation}
    \begin{enumerate}
        
        \item numero ()
        \item \begin{equation}
            \frac{c+0}{0} = (\pm \infty )
        \end{equation}
        \item \begin{equation}
            \frac{0}{0}
        \end{equation} Factorizar y Calcular (propiedades de limites)
    \end{enumerate}
               
    \section{Clase 5}
    \textbf{Algebra de Limites}: supongamos que \begin{equation}
        \begin{aligned}
        && \lim_{x\rightarrow a}f(x) = L1 \ y \\
        && \lim_{x\rightarrow a}g(x) = L2 \ y \ c \epsilon \mathcal{R}
        \end{aligned}
    \end{equation}

    \begin{enumerate}
        \item \begin{equation}
            \lim_{x\rightarrow a} (f(X)+g(x)) = \lim_{x\rightarrow a}f(x) + 
            \lim_{x\rightarrow a} g(x) = L1 +L2
        \end{equation}
        \item \begin{equation}
            \lim_{x \rightarrow a}C\cdot f(x) = C\cdot \lim_{x\rightarrow a}f(x)
            = C\cdot L1
        \end{equation}
        \item \begin{equation}
            \lim_{x\rightarrow a} (f(x)\cdot g(x)) = 
            \lim_{x \rightarrow a} f(x) \cdot \lim_{x \rightarrow a} g(x) 
            = L1 \cdot L2
        \end{equation}
        \item \begin{equation}
            \lim_{x\rightarrow a} \frac{f(x)}{g(x)} = \frac{\lim_{x\rightarrow a}f(x)}{\lim_{x\rightarrow a}g(x)} 
            \ si \ \lim_{x\rightarrow a}g(x) \neq 0
        \end{equation}
    \textbf{Ejemplo}
    \begin{enumerate}
        \item \begin{equation} \begin{aligned}
            \lim_{x\rightarrow 2} (3x^2 -8)\cdot(\sqrt[]{3x+3}) \\
            \lim_{x\rightarrow 2} (3x^2 -8)\ =4 \ y \ \lim_{x\rightarrow 2}\sqrt[]{3x+3} \ =3
        \end{aligned}
        \end{equation} \begin{center}
            Ambos existen. Finalmente L1 $\cdot$ L2 = 3 $\cdot$ 4 = 12
        \end{center}
    \end{enumerate}
    \textbf{Veamos}
    \begin{equation}
        \begin{aligned}
            \lim_{x\rightarrow 1+} \frac{f(x)}{f(x)- g(x)} =
            \frac{\lim_{x\rightarrow 1+}f(X)}{\lim_{x\rightarrow 1+}f(x)- g(x)} = 
            \frac{2}{2-(-2)} = \frac{1}{2}
        \end{aligned}
    \end{equation} (continua...)
    \item 
    \begin{equation}
        \begin{aligned}
            \lim_{x\rightarrow a}[f(x)]^n = [\lim_{x\rightarrow a}f(x)]^n = L_{1}^n
        \end{aligned}
    \end{equation}
    \item
    \begin{equation}
        \begin{aligned}
            \lim_{x\rightarrow a} \sqrt[n]{f(x)} = \sqrt[n]{\lim_{x\rightarrow a}f(x)}
            = \sqrt[n]{L_{1}}
        \end{aligned}
    \end{equation}
    \item Si P(x) es un polinomio \begin{equation}
        \begin{aligned}
            \lim_{x\rightarrow a}P(x) = P(a)
        \end{aligned}
    \end{equation}
    \item \begin{equation}
        \lim_{x\rightarrow a} C = C
    \end{equation}
\end{enumerate}
\textbf{Ahora}: \newline
Suponga que \begin{equation}
    \lim_{x\rightarrow 0}\frac{f(x)}{x} = 6
\end{equation} Calcule:
\begin{enumerate}
    \item \begin{equation}
        \begin{aligned}
            \lim_{x\rightarrow 0 } f(x) &&\\
            \lim_{x\rightarrow 0}f(x) = \lim_{x\rightarrow 0} f(x) \cdot \frac{x}{x} &&
            = \lim_{x\rightarrow 0}\frac{f(x)}{x}\cdot x \\
            = \lim_{x\rightarrow 0}\frac{f(x)}{x}\cdot \lim_{x\rightarrow 0} x = 6 \cdot 0 = 0 
        \end{aligned}
    \end{equation}
    \item \begin{equation}
        \begin{aligned}
            \lim_{x\rightarrow 0 }\frac{(f(x))^2}{x^2} = \lim_{x\rightarrow 0 }(\frac{f(x)}{x})^2 \\
            (\lim_{x\rightarrow 0 }\frac{f(x)}{x})^2 = 6^2 = 36
        \end{aligned}
    \end{equation}
    \item \begin{equation}
        \begin{aligned}
            \lim_{x\rightarrow 0}\frac{f(x)}{\sqrt[]{x}} = 
            \lim_{x\rightarrow 0} \frac{f(x)}{\sqrt[]{x}} \cdot 
            \frac{\sqrt[]{x}}{\sqrt[]{x}} \\
            \lim_{x\rightarrow 0} \frac{f(x)}{x} \cdot \sqrt[]{x}
            = \lim_{x\rightarrow 0 }\frac{f(x)}{x}\cdot \lim_{x\rightarrow 0} \sqrt[]{x}= \\
            = 6\cdot 0 = 0
        \end{aligned}
    \end{equation}
\end{enumerate}
\subsection{Cambio de variable}
\begin{center}
    \begin{equation}
        \begin{aligned}
            \lim_{x\rightarrow a }f(g(x))
        \end{aligned}
    \end{equation}
\end{center}Sea $\mathcal{U}$ = g(x) \begin{enumerate}
    \item Si x$\rightarrow$a $\Longrightarrow$  $\mathcal{U}\rightarrow$ g(x) (Nueva Tendencia)
    \item \begin{equation}
        \lim_{x\rightarrow a}f(g(x)) = \lim_{\mathcal{U}\rightarrow g(x)}f(\mathcal{U})
    \end{equation}
    \item Calcule: \begin{equation}
        \begin{aligned}
            \lim_{x\rightarrow 0}\frac{\sqrt[]{x+1}-1}{x}
        \end{aligned}
    \end{equation} Sea $\mathcal{U}$ = $\sqrt[]{x+1}$ $\Longleftrightarrow $
    $\mathcal{U}=$ x+1 $\Longleftrightarrow$ x = $\mathcal{U}^2$-1
    \begin{enumerate}
        \item Si x $\rightarrow$ 0 $\Longrightarrow \mathcal{U} \rightarrow$ 1
        \item \begin{equation}
            \begin{aligned}
                \lim_{x\rightarrow 0} \frac{\sqrt[]{x+1} -1}{x} = 
                \lim_{\mathcal{U}\rightarrow 1}\frac{\mathcal{U}-1}{\mathcal{U}^2 -1} \\
                \lim_{\mathcal{U}\rightarrow 1} \frac{\mathcal{U}-1}{(\mathcal{U}-1)(\mathcal{U}+1)} \\
                \lim_{\mathcal{U}\rightarrow 1} \frac{1}{\mathcal{U}+1} = \frac{1}{2}
            \end{aligned}
        \end{equation} Asi \begin{equation}
            \begin{aligned}
                \lim_{x\rightarrow 0}\frac{\sqrt[]{x+1}-1}{x} = \frac{1}{2}
            \end{aligned}
        \end{equation}
    \end{enumerate}
    \item Calcule \begin{equation}
        \begin{aligned}
            \lim_{x\rightarrow 1}\frac{\sqrt[3]{x}-1}{\sqrt[4]{x}-1}
        \end{aligned}
    \end{equation} Sea: $\mathcal{U}=\sqrt[12]{x}\Longleftrightarrow x = \mathcal{U}^{12}$
\end{enumerate}
\subsection{Teorema del Sanwich}
Sea f(x), g(x) y h(x) funciones tal que: f(x) $\leq$ g(x) $\leq$ h(x), cuando x se acerca a "a" 
y ademas \begin{equation}\begin{aligned}
    \lim_{x\rightarrow a}f(x) = L = \lim_{x\rightarrow a}h(x) \\
 Entonces, \
    \lim_{x\rightarrow a} g(x) = L
\end{aligned}
\end{equation}

\end{document} 

